\section{RNA-Seq}
\subsection{技术简介}
\begin{frame}[label=current]
  \frametitle{转录组学 | RNA-Seq | 简介}
  \begin{block}{RNA-Seq}
 RNA测序(RNA Sequencing,简称RNA-Seq,也被称为全转录物组鸟枪法测序,Whole Transcriptome Shotgun Sequencing,简称WTSS)是基于第二代测序技术的转录组学研究方法。RNA测序是使用第二代测序的能力,在给定时刻从一个基因组中,揭示RNA的存在和数量的一个快照的技术。
  \end{block}
\end{frame}

\begin{frame}[label=current]
  \frametitle{转录组学 | RNA-Seq | 简介}
  \begin{block}{RNA-Seq}
首先提取生物样品的全部转录的RNA,然后反转录为cDNA后进行二代高通量测序,在此基础上进行片段的重叠组装,从而可得到一个个的转录本。进而可以形成对该生物样品当前发育状态的基因表达状况的全局了解。进一步说,若和下一阶段的生物样品的RNA-Seq转录组进行比较,则可以得到全部的(在转录层面)基因表达的上调及下调——这就形成了表达谱,针对关键基因则可以形成你要想要的通路(pathway)的构建。
  \end{block}
\end{frame}

\begin{frame}[label=current]
  \frametitle{转录组学 | RNA-Seq | 简介}
  \begin{block}{技术}
下一代基因测序之前,对转录物组学和基因表达的研究主要基于基因表达芯片(微阵列),后者包含数以千计用于探测靶向序列的DNA探针,可以得到所有表达出转录物的表达谱。\\
\vspace{1em}
基因表达芯片之后,基因表达的系列分析(SAGE)是主要的基因分析技术。
  \end{block}
\end{frame}

\begin{frame}[label=current]
  \frametitle{转录组学 | RNA-Seq | 简介}
  \begin{block}{RNA-Seq}
相较于一个静态的染色体而言,细胞内的转录物组是一个处于不断变化的动态过程。随着现在的下一代基因测序(NGS)技术的发展,使得可测得的DNA碱基覆盖面增加且样本输出的吞吐量增大。有助于对细胞内RNA转录物进行测序,提供包括选择性剪接的转录、转录后的改变、基因融合、突变/SNPs以及基因表达量改变等细节。,RNA测序不仅能检测mRNA的转录,还能观测到包括总RNA和小RNA(miRNA、tRNA和核糖体RNA)在内不同尺度的RNA表达谱。RNA测序还能用来确定外显子/内含子的边界,修正之前注释的5'和3'端基因边界。未来的RNA测序研究还包括观察感染时细胞传导路径的变化和癌症中不同基因表达程度。
  \end{block}
\end{frame}

\begin{frame}[label=current]
  \frametitle{转录组学 | RNA-Seq | 简介}
\end{frame}

\subsection{数据分析}
