% \documentclass[UTF8,table]{ctexbeamer}
\documentclass[UTF8,table]{beamer}
\usepackage[fontset=adobe]{ctex}

\mode<presentation>
{
  \usetheme{Madrid}
}
\setbeamercolor{bgcolor}{fg=yellow,bg=cyan}
% 使所有隐藏的文本完全透明、动态,而且动态的范围很小
\beamertemplatetransparentcovereddynamic
% 使itemize环境中变成小球,这是一种视觉效果
\beamertemplateballitem
% 为所有已编号的部分设置一个章节目录,并且编号显示成小球
\beamertemplatenumberedballsectiontoc
% 将每一页的要素的要素名设成加粗字体
\beamertemplateboldpartpage
% item逐步显示时,使已经出现的item、正在显示的item、将要出现的item呈现不同颜色
\def\hilite<#1>{
 \temporal<#1>{\color{gray}}{\color{blue}}
    {\color{blue!25}}
}

% 设定英文字体
% \usepackage[no-math]{fontspec}
\setmainfont{Times New Roman}
\setsansfont{Arial}
\setmonofont{Courier New}
% % 设定中文字体
% \usepackage[BoldFont,SlantFont,CJKchecksingle,CJKnumber]{xeCJK}
% \setCJKmainfont[BoldFont={Adobe Heiti Std},ItalicFont={Adobe Kaiti Std}]{WenQuanYi Micro Hei}
% \setCJKsansfont{Adobe Heiti Std}
% \setCJKmonofont{Adobe Fangsong Std}
% \punctstyle{hangmobanjiao}
% \defaultfontfeatures{Mapping=tex-text}
% \usepackage{xunicode}
% \usepackage{xltxtra}
% \XeTeXlinebreaklocale "zh"
% \XeTeXlinebreakskip = 0pt plus 1pt minus 0.1pt

\renewcommand{\today}{\number\year 年 \number\month 月 \number\day 日}

\usepackage{hyperref}
\hypersetup{xetex,bookmarksnumbered=true,bookmarksopen=true,pdfborder=1,breaklinks,colorlinks,linkcolor=cyan,filecolor=black,urlcolor=blue,citecolor=green}

%分栏
\usepackage{multicol}

% 插入图片
\usepackage{graphicx}
\graphicspath{{figures/}}
% 图文混排
% \usepackage{floatflt}

% 可能用到的包
% \usepackage{amsmath,amssymb}
% \usepackage{setspace}
% \usepackage{colortbl,xcolor}

%五角星
% \usepackage{MnSymbol}

%去除图表标题中的figure等
\usepackage{caption}
\captionsetup{labelformat=empty,labelsep=none}

\usepackage{tabu}
%表格自动换行
\usepackage{tabularx} 
\usepackage{multirow}

%罗马数字
\makeatletter
\newcommand{\rmnum}[1]{\romannumeral #1}
\newcommand{\Rmnum}[1]{\expandafter\@slowromancap\romannumeral #1@}
\makeatother

%插入源代码
\usepackage{listings}
\lstset{
  language=bash,                  % 程序语言名称:TeX, Perl, R, sh, bash, Awk
  basicstyle=\normalsize\tt,      %\tt指monospace字体族,程序源代码使用此族字体表示更加美观
  numbers=left,                   % 行号位置(左侧)
  numberstyle=\small,             % 行号字体的字号
  stepnumber=1,                   % 行号的显示步长
  numbersep=5pt,                  % 行号与代码间距
  backgroundcolor=\color{white},  % 背景色;需要 \usepackage{color}
  showspaces=false,               % 不显示空格
  showstringspaces=false,         % 不显示代码字符串中的空格标记
  showtabs=false,                 % 不显示 TAB
  tabsize=4, 
  frame=shadowbox,                % 把代码用带有阴影的框圈起来
  captionpos=b,                   % 标题位置
  breaklines=true,                % 对过长的代码自动断行
  breakatwhitespace=false,        % 断行只在空格处
  extendedchars=false,            % 解决代码跨页时,章节标题,页眉等汉字不显示的问题
  %escapeinside={\%*}{*},         % 跳脱字符,添加注释,暂时离开 listings 
  %escapeinside=``,
  commentstyle=\color{red!50!green!50!blue!50}\tt,  %浅灰色的注释
  rulesepcolor=\color{red!20!green!20!blue!20},     %代码块边框为淡青色
  keywordstyle=\color{blue!70}\bfseries\tt,         %代码关键字的颜色为蓝色,粗体
  identifierstyle=\tt,
  stringstyle=\tt,                % 代码字符串的特殊格式
  keepspaces=true,
  breakindent=1em,
  %breakindent=22pt,
  %breakindent=4em,
  breakautoindent=true,
  flexiblecolumns=true,
  aboveskip=1em,                  %代码块边框
  xleftmargin=2em,
  xrightmargin=2em
}

